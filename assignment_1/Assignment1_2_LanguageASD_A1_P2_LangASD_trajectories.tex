\documentclass[]{article}
\usepackage{lmodern}
\usepackage{amssymb,amsmath}
\usepackage{ifxetex,ifluatex}
\usepackage{fixltx2e} % provides \textsubscript
\ifnum 0\ifxetex 1\fi\ifluatex 1\fi=0 % if pdftex
  \usepackage[T1]{fontenc}
  \usepackage[utf8]{inputenc}
\else % if luatex or xelatex
  \ifxetex
    \usepackage{mathspec}
  \else
    \usepackage{fontspec}
  \fi
  \defaultfontfeatures{Ligatures=TeX,Scale=MatchLowercase}
\fi
% use upquote if available, for straight quotes in verbatim environments
\IfFileExists{upquote.sty}{\usepackage{upquote}}{}
% use microtype if available
\IfFileExists{microtype.sty}{%
\usepackage{microtype}
\UseMicrotypeSet[protrusion]{basicmath} % disable protrusion for tt fonts
}{}
\usepackage[margin=1in]{geometry}
\usepackage{hyperref}
\hypersetup{unicode=true,
            pdftitle={Assignment 1 - Language Development in ASD - part 2},
            pdfauthor={Riccardo Fusaroli},
            pdfborder={0 0 0},
            breaklinks=true}
\urlstyle{same}  % don't use monospace font for urls
\usepackage{graphicx,grffile}
\makeatletter
\def\maxwidth{\ifdim\Gin@nat@width>\linewidth\linewidth\else\Gin@nat@width\fi}
\def\maxheight{\ifdim\Gin@nat@height>\textheight\textheight\else\Gin@nat@height\fi}
\makeatother
% Scale images if necessary, so that they will not overflow the page
% margins by default, and it is still possible to overwrite the defaults
% using explicit options in \includegraphics[width, height, ...]{}
\setkeys{Gin}{width=\maxwidth,height=\maxheight,keepaspectratio}
\IfFileExists{parskip.sty}{%
\usepackage{parskip}
}{% else
\setlength{\parindent}{0pt}
\setlength{\parskip}{6pt plus 2pt minus 1pt}
}
\setlength{\emergencystretch}{3em}  % prevent overfull lines
\providecommand{\tightlist}{%
  \setlength{\itemsep}{0pt}\setlength{\parskip}{0pt}}
\setcounter{secnumdepth}{0}
% Redefines (sub)paragraphs to behave more like sections
\ifx\paragraph\undefined\else
\let\oldparagraph\paragraph
\renewcommand{\paragraph}[1]{\oldparagraph{#1}\mbox{}}
\fi
\ifx\subparagraph\undefined\else
\let\oldsubparagraph\subparagraph
\renewcommand{\subparagraph}[1]{\oldsubparagraph{#1}\mbox{}}
\fi

%%% Use protect on footnotes to avoid problems with footnotes in titles
\let\rmarkdownfootnote\footnote%
\def\footnote{\protect\rmarkdownfootnote}

%%% Change title format to be more compact
\usepackage{titling}

% Create subtitle command for use in maketitle
\newcommand{\subtitle}[1]{
  \posttitle{
    \begin{center}\large#1\end{center}
    }
}

\setlength{\droptitle}{-2em}

  \title{Assignment 1 - Language Development in ASD - part 2}
    \pretitle{\vspace{\droptitle}\centering\huge}
  \posttitle{\par}
    \author{Riccardo Fusaroli}
    \preauthor{\centering\large\emph}
  \postauthor{\par}
      \predate{\centering\large\emph}
  \postdate{\par}
    \date{July 7, 2017}


\begin{document}
\maketitle

\section{Language development in Autism Spectrum Disorder
(ASD)}\label{language-development-in-autism-spectrum-disorder-asd}

Background: Autism Spectrum Disorder is often related to language
impairment. However, this phenomenon has not been empirically traced in
detail: i) relying on actual naturalistic language production, ii) over
extended periods of time.

We therefore videotaped circa 30 kids with ASD and circa 30 comparison
kids (matched by linguistic performance at visit 1) for ca. 30 minutes
of naturalistic interactions with a parent. We repeated the data
collection 6 times per kid, with 4 months between each visit. We
transcribed the data and counted: i) the amount of words that each kid
uses in each video. Same for the parent. ii) the amount of unique words
that each kid uses in each video. Same for the parent. iii) the amount
of morphemes per utterance (Mean Length of Utterance) displayed by each
child in each video. Same for the parent.

This data is in the file you prepared in the previous class.

NB. A few children have been excluded from your datasets. We will be
using them next week to evaluate how good your models are in assessing
the linguistic development in new participants.

We then want to test the language trajectory of child and parent over
time.

This RMarkdown file is structured in the following way:

\begin{enumerate}
\def\labelenumi{\arabic{enumi}.}
\tightlist
\item
  The exercises: read them carefully. Under each exercise you will have
  to write your answers, once you have written and run the code. This is
  the part that you have to directly send to the teachers.
\item
  An (optional) guided template full of hints for writing the code to
  solve the exercises. Fill in the code and the paragraphs as required.
  Then report your results under the exercise part.
\item
  In exercise 4 you will be asked to create the best possible model of
  language development in TD and ASD children, picking and choosing
  whatever additional variables you want from the dataset. Next time,
  the models produced by the different groups will compete against each
  other to see who can produce the best model, so choose carefully!
\end{enumerate}

You will have to have a github repository for the code and send the
answers to Malte and Riccardo without code (but a link to your
github/gitlab repository). This way we can check your code, but you are
also forced to figure out how to report your analyses :-)

Remember to submit only your findings, and not just the code. To do this
you can either - Write your answers in a separate document - Write your
answers in the template, but tell rstudio not to print the code chunks
when you knit it with the chunk option include=FALSE

\begin{verbatim}
## [1] 11
\end{verbatim}

To re-iterate one more time: Hand in a document with your findings but
without code (html or pdf or word) and a link to your github/gitlab with
your Rmd file.

N.B. The following lines are a summary of the questions to be answered,
the step-by-step instructions and tips are in the template.

\subsection{Exercise 1) Preliminary Data
Exploration}\label{exercise-1-preliminary-data-exploration}

Describe the participant samples in the dataset (e.g.~by diagnosis, age,
etc.). Do you think the two groups are well balanced? If not, what do
you think was the reason?

Participants samples are well balanced according to - Diagnosis - almost
the same number of participants, 10 more TD participants - Age -
Types\_MOT - spread all over but are slightly different, ASD group peaks
at 350 and TD group peaks at 400

not well balanced - MLU\_MOT - mothers of TD children has slightly
higher mean than ASD group - CHI\_MLU - TD group says much more words
while ASD group says the most from 1-1,5 - types\_CHI - ASD group says
much less(up to 50) unique words in comparison to TD group (up to 200) -
token\_CHI - TD group said much more words (up to 800) while ASD up to
200

\subsubsection{Exercise 2) Children learning language: the effects of
time and
ASD}\label{exercise-2-children-learning-language-the-effects-of-time-and-asd}

Describe linguistic development in TD and ASD children in terms of Mean
Length of Utterance (MLU)?

In our model, we used child's MLU as an a dependent variable, visit and
diagnosis as independent variables. We also included random effects,
specifically subject as random intercept and visit as random slope.
After analysis, we found the visit as a significant predictor of child's
MLU , β = 0.23,(SE = 0.02), t = 9.516, p \textless{} .0001. However,
diagnosis did not turn out significant, β = 0.29 ,(SE = 0.15), t = 1.91,
p \textgreater{} .05. The result indicates that ASD children's
linguistic performance is not different from non-ASD children. But it
also suggests that overall children MLU is changing with each visit.
Aditionally, we calcualted r2 for our model, which was following:
R2m=0.219 and R2c = 0.803. This indicates that our model with random
effects explains significatly more variance but we cannot further
generalize it.

\subsubsection{Exercise 3) Child directed speech as a moving
target}\label{exercise-3-child-directed-speech-as-a-moving-target}

Describe how parental use of language changes over time in terms of MLU.
What do you think is going on?

According to our analysis, mothers' MLU changes depending on Diagnosis,
β = 0.50,(SE = 0.11), t = 4.42, p \textless{} .0001. And diagnosis is a
significant predictor as well as time (in our case Visit), β = 0.12, (SE
= 0.02), t = 6.60, p \textless{} .0001. This might suggest that knowing
of diagnosis changes mothers' speech accordingly. Moreover, the R2 for
the model: R2m = 0.230 and R2c= 0.676, similarly suggests that including
of random effects helps to explain more variance but cannot be
generalized.

\subsubsection{\texorpdfstring{Exercise 4) Looking into ``individual
differences'' (demographic, clinical or cognitive
profiles)}{Exercise 4) Looking into individual differences (demographic, clinical or cognitive profiles)}}\label{exercise-4-looking-into-individual-differences-demographic-clinical-or-cognitive-profiles}

The dataset contains some additional variables characterizing the kids'
cognitive and clinical profile: ADOS (autism severity), MSEL EL
(Expressive Language, that is, verbal IQ, or linguistic skills at first
visit as assessed by a psychologist using Mullen Scales of Early
Learning), MSEL VR (Visual Reception, used as a proxy for non verbal IQ
at first visit), Age, Gender, Ethnicity. Would it make sense to add any
of them to your model of linguistic trajectories? Create the best
possible model (the one that best explain the data, with MLU as
outcome). Next time your model will be tested on new participants, and
we will proclaim a winner. Describe your strategy to select the best
models (how did you choose the variables to include?) and send the code
to Riccardo and Celine.

In order to identify best model, that describes our data, we have chosen
as independant variable VISIT to predict Child MLU because we are
interested in linguitic trajectory. We belive that we can't use
additional clinical variables (e.g.~noverbalIQ or ADOS), because they
were collected not every visit, thus we don't have enough data.
Additionally, we see no difference in variable AGE and Visit, so it
makes no sence to add Age as IV. In our opinion, ethnicity and gender
plays no role. We believe that child MLU is calculated from overall
amount of words, which is token\_CHI variable. Therefore it is not
reasonable to make token\_CHI as predictor. We did correlation analysis
between CHI\_MLU and types\_chi and we discovered strong correlation. In
our opinion trying to predict one index of liguistic performance, which
is Child MLU, by another index of linguistic performance doesn't make
much sense, because they belong to one category and have strong
correlation. Also if we do so, our main predictor, which is VISIT
becomes not significant. However, our main priority is to see linguistic
trajectory over time. Originally the idea was that Diagnosis could be a
significant predictor, however, earlier in the code it was clear that
diagnosis is not significant.

\subsubsection{\texorpdfstring{{[}OPTIONAL{]} Exercise 5) Comment on how
the three linguistic variables measure linguistic performance (the
so-called ``construct validity'' of the measures). Do they express the
same
variance?}{{[}OPTIONAL{]} Exercise 5) Comment on how the three linguistic variables measure linguistic performance (the so-called construct validity of the measures). Do they express the same variance?}}\label{optional-exercise-5-comment-on-how-the-three-linguistic-variables-measure-linguistic-performance-the-so-called-construct-validity-of-the-measures.-do-they-express-the-same-variance}

According to aforementioned answer, we believe that three lingusitic
variables express the same variance as they are strongly correlated and
dependent on each other.


\end{document}
